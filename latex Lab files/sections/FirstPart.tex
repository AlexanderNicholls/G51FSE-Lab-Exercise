\section{First Member}
This is the section dedicated to one of the team members, and it should be written individually . It can include a range of things; first subsection is a space for you to point out the strengths and weaknesses of the module, including complaints about the module coordinator Max Wilson. The second section should have a selfie image with Max! The last part of it is the most important one. You will need to write a paragraph about what you have learned in this module. You can write it in \textbf{Bold} if you want or you can use other fonts. 

Please do not forget:
\begin{itemize}
	\item First paragraph should have your comments about the module
	\item Second one, a selfie img with Max
	\item Last one, what you learned in this module.
\end{itemize}

\subsection{Comments about the module}
G51FSE Software Engineering turned out to be somewhat different than what I had initially thought. The content in the module was drastically different to the modules taken in the first semester, all of which were packed with either maths or computer knowledge. Creating text analyses and personas was a shock to the system, as there was no real 'right answer.' At first, since it was all so new, it was somewhat stressful to figure out what exactly was expected of me, having little experience with this kind of work; I was much more used to work being logical and with a clear 'right answer.' This got better over time though, as more experience with this material made it easier to understand and much less stressful to work with. The group work aspect of this module also made it easier to adjust as the workload was shared and we could discuss ideas freely within our group.

\subsection{Selfie with Max}

To include an image, you will need to remove the comments from the code below, place an image in the main folder, and do not forget to put the name of the image instead of ImgName. 

%\begin{figure}[h]
%\caption{Selfie with Max}
%\centering
%\includegraphics[width=0.5\textwidth]{ImgName.jpg}
%\label{fig:selfie}
%\end{figure}

You can then use the label of the figure to reference it later with the command ${\backslash}ref$. you can comment out the next line to see an example of how it works.

% My selfie with Max is in  Figure~\ref{fig:selfie}.

\subsection{What I have learned in this module}
This is some random text.
This is the first subsection. Do not forget to include the best parts about the module as well as what you did not like about Max Wilson during the term.
